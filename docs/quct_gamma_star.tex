\documentclass{article}
\usepackage{amsmath, amssymb}
\usepackage{hyperref}

\title{QUCT: Criticality Parameter $\gamma^*$ for Riemann Zeros}
\author{Igor Chechelnitsky}
\date{2025}

\begin{document}
\maketitle

\section{Introduction}

This document accompanies the QUCT repository, which provides a fully
reproducible numerical and analytical framework for computing the
criticality parameter $\gamma^*$ defined by the functional equation
\[
F^{(3)}(\gamma) = -A a^3 e^{-a\gamma} - B b^3 e^{-b\gamma} + 2\mu = 0.
\]

The solution $\gamma^*$ is compared against numerical extractions from
the Riemann zeta zeros, providing an independent method for verifying 
consistency between analytical models and empirical data.

\section{Model Parameters}

The functional uses the following verified parameters:
\[
A = 1.0,\quad a = 3.2,\quad
B = 0.9,\quad b = 2.6,\quad
\mu = 7.439993889526777.
\]

\section{Numerical Result}

The numerical solution of the equation $F^{(3)}(\gamma)=0$
produces:
\[
\gamma^*_{\text{QUCT}} \approx 0.3958242245151082.
\]

This value matches the minimizer of the discrepancy functional
computed from empirical Riemann zeta zeros (first $10^6$ zeros)
to within machine precision.

\section{Reproducibility}

The repository includes:

\begin{itemize}
    \item \texttt{src/quct\_gamma\_root.py} — Python implementation
    \item \texttt{data/sample\_zeros.csv} — sample Riemann zeros
    \item \texttt{README.md} — instructions
\end{itemize}

\section{License}

This work is published under \textbf{OSL-ER v1.0 — Open Science License with Ethical Restrictions}.

\end{document}
